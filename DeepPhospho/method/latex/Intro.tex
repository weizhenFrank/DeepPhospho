Taking advantage of the vast number of peptides retention time and tandem mass spectra data,
we report a new deep learning tool termed DeepPhospho. DeepPhospho could learn and predict the chromatographic retention time and the fragment ion intensity of any peptide with extremely high quality. 
We use the LSTM + Transformer as its architecture; the LSTM could learn a good amino acid representation for the downstream Transformer module. By exploiting self-attention, the Transformer module could capture the difference of amino acids much more precisely. More importantly, the self-attention mechanism could potentially force model to attend the pair of b ion and y ion as these two ions are broke down at the same time in mass spectrometer.

In the supplementary, we show the superiority of LSTM + 
Transformer compared to solely Transformer or LSTM. Based on the observation that the Transformer module needs 
the good initial embedding of amino acid, we wonder that if whether or not a convolutional neural network (CNN) 
could replace the LSTM module to learn a good representation of amino acid, and the results show that the LSTM module is better than the CNN module configured before the Transformer. In further, we want to know a purely deep CNN whether if better than the LSTM + Transformer; however, results show a deep CNN could not generalize well compared with LSTM + Transformer architecture. 
To the best of our knowledge, DeepPhospho is the first work to utilize the Transformer on the peptide spectrum prediction
though it has been prevalently used in the natural language processing domain.
There have been several tools that successfully utilized deep learning in RT and Ion intensity prediction, such as DeepRT\cite{ma2018improved} and pDeep2\cite{zeng2019ms}.
DeepRT use the capsule network (CapsNet) \cite{sabour2017dynamic} 
model to predict RT. However, the DeepRT did not explain very well for the choice of CapsNet, and CapsNet's structure cannot fully utilize the sequence's characteristics. 
pDeep2 just use the bi-directional LSTM model to predict the ion intensity, and it does not explicitly take the relations of amino acids into the consideration of model design.
We demonstrate the merits of DeepPhospho on a number of challenging examples and provide the scientific community 
with ready-to-use tools.