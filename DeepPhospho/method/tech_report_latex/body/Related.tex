\subsection{Retention Time Prediction}
For retention time prediction, previous efforts to predict peptide RTs are mainly based on retention coefficients (Rc) of amino acids,
while SSRCalc~\cite{guo1986prediction} is the most popular Rc-based RT predictor. Rc is a parameter to appraise the contribution of an individual amino acid to peptide RT, and the sum of all the Rcs of amino acids in a peptide could serve for RT estimation.
Additional factors such as peptide length, charge, and helicity are also considered during peptide RT prediction.
Several predictors based on Rc and other measurable factors have been proposed and reported in some studies. For example, Elude~\cite{moruz2010training,moruz2012chromatographic} and GPTime~\cite{maboudi2017uncertainty} developed from support vector machine (SVM) and Gaussian process
regression employed Rcs learned from data sets and can also provide RT prediction for post-translationally modified (PTM) peptides. However, these tools could not predict RT quite accurate that it is well recognized
that these measurable factors are far not enough to fully illustrate the physicochemical properties of peptides and the complex interactions between peptides and stationary phase. Hence, in the domain of peptide RT prediction, there is still large room for improvement.

Deep learning, an advanced machine learning method, has shown extraordinary capability to learn complex relationships from large-scale data. There have been several tools that successfully utilized deep learning in RT prediction, such as DeepRT~\cite{ma2018improved}, and Prosit~\cite{gessulat2019prosit}. DeepRT uses the capsule network (CapsNet)~\cite{sabour2017dynamic} model and could foresaw the RTs for the peptides at different modification status included as oxidation of methionine, phosphorylation of
serine, threonine, tyrosine and at varied experiment conditions.
Prosit utilizes the LTSM model and like DeepRT, it could predict the retention time given the peptide sequence. However, the DeepRT did not explain very well for the choice of CapsNet, and it did not fully consider the sequence's characteristics. 

\subsection{Ion Intensity Prediction}
Investigation of the peptide fragmentation is valuable both in theory and in practice and there are some works about the prediction of theoretical MS/MS spectra of peptides, including kinetic model-based methods and machine learning based methods. MassAnalyzer~\cite{zhang2004prediction, zhang2005prediction} and MS-Simulator~\cite{sun2012ms,wang2015openms} are two major kinetic model-based tools designed based on the mobile proton hypothesis with some basic assumptions. But the kinetic model cannot consistently be used to model the peptide fragmentation under different spectrometry conditions. 
PeptideART is a pure machine learning based tool which transforms the theoretical spectrum prediction as a classification problem, and it uses a shallow feed-forward neural network~\cite{arnold2006machine,li2011accuracy} to learn the probability of the occurrence of each peak.
It has been widely recognized that a good spectrum prediction method can boost the identification of peptides. However, peptide fragmentation is very complex to predict and to handle the complexity of peptide fragmentation, more powerful algorithms such as deep learning could be considered.

pDeep~\cite{zhou2017pdeep}, a deep learning-based method based on LSTM model to predict the intensity distribution of product ions of a peptide. pDeep can
work well in predicting kinds of spectra where it achieves $>$0.9 median PCCs (Pearson correlation coefficient). And its performance is significantly better than kinetic model-based MassAnalyzer and MS-Simulator as well as the machine learning-based PeptideART. After pDeep, the modified version of pDeep, called pDeep2~\cite{zeng2019ms} is proposed. pDeep2 is spectrum predictor for modified peptides like phosphorylated peptides based on the deep learning model. It uses the transfer-learning technique to transfer pDeep model parameters on non-modified to the prediction of modified peptide's spectrum.