\subsection{RT Dataset}
The sequence is represented by the symbol of amino acids such as L, K, M, etc., typically 7-50 
in length. Specially, we use 1 to represent the oxidation of methionine (M), and we use 2,3,4 to represent the phosphorylation of serine(S), threonine(T), tyrosine(Y), respectively. And those representation is also used in the Ion Intensity task.
Retention time (RT) is a measure of the time taken for a solute to pass through a chromatography column.
It is calculated as the time from injection to detection. 

The RT datasets are comprised of pairs of 
\( \{X, y\} \). $X:= \{ <x_1, x_2, x_3,\dots, x_n> \}$, $x_i$ 
is amino acid,
and \( y \) is the retention time. For building the virtual library, we split the dataset into train : validation = 9 : 1, 
selecting the best model on validation set; for model comparison, the dataset is split into train : validation : test
= 8 : 1 : 1, reporting results on the test set.

As the retention time is distributed in the real-world unit, such as minutes or seconds, we scale each dataset by its max and min of retention time to 0 - 1 by the following formula. To cover all RT dataset distributions, we set the max(RT) as 200, and min(RT) as -100. 

In further, we support the peptide with N-terminal acetyl modification. We use the * symbol to indicate modification, @ to indicate no modification. We pad all sequences to the length of the longest sequence in the dataset to form a matrix feeding into the neural network. 

\subsection{Ion Intensity Dataset}
Like RT dataset, the ion intensity datasets are comprised of pairs of 
\( \{X, y\} \). $X := \{ <x_1, x_2, x_3,\dots, x_i, \dots, x_n, +q> \}$, $x_i$ is amino acid, $+q$ 
is the charge carried by the
peptide sequence before it is fragmented in the mass spectrometer. And \( y \) is the spectrum of the peptide. Each y is composed of pairs of key and value.
The key is the ion's name, such as y2+1, b6+2, and the value is their corresponding raw intensity.
We divide each intensity by the maximum of the intensities within a peptide sequence to normalize each intensity into 0-1. As kinds of ions in the dataset is severely imbalanced, we only select the 8 types of ions same as pdeep2\cite{zeng2019ms}, that is 
b(y)i+1-noloss, b(y)i+2-noloss, b(y)i+1-1,H3PO4 and b(y)i+2-1,H3PO4, i indicating the site of b(y)ion, 
to feed the neural network and predict those 8 types of ion intensity. Those ion intensities are formed as the matrix of shape 8 * length of the sequence (illustrated in the supplementary). 
There are two types of fragment ion intensity values that have no contribution in the loss calculation and are removed in the prediction. One is that related to padding, like y20 for a 7-mer; the other is related to the phosphorylation site. For example, the phosphorylation site is located in the b5, then the b1,b2, b3, and b4 ions cannot lose phosphate so that the ion intensity with phosphate loss must be 0. This ignorance of impossible phosphorylation site potentially help model to learn the implicitly rule of phosphorylation, benefiting the prediction accuracy of intensity of ion with phosphate loss. Otherwise, the only way for model to learn this rule is from the ion intensity data which would be much more inefficient than the injecting the prior knowledge directly to the model learning. 
The data split, N-terminal acetyl modification indicator and padding operation are the same as the RT dataset.
