\subsection{Retention Time Prediction}
For retention time prediction, efforts to date to predict peptide RTs are mainly based on retention coefficients (Rc) of amino acids,
while SSRCalc~\cite{guo1986prediction} is the most popular Rc-based predictor. Rc is a parameter to appraise the contribution of an individual amino
acid to peptide RT, and the sum of all the Rcs of amino acids in a peptide could serve for RT estimation.
Additional factors such as peptide length, charge, and helicity are also considered during peptide RT prediction.
Several predictors based on Rc and other measurable factors have been proposed and reported in some studies. For example, Elude~\cite{moruz2010training,moruz2012chromatographic} and GPTime~\cite{maboudi2017uncertainty} developed from support vector machine (SVM) and Gaussian process
regression employed Rcs learned from data sets and can also provide RT prediction for post-translationally modified (PTM) peptides.
All these tools produced RT with $R^2$ values of less than 0.965 on various data sets. On the other hand, it is well recognized
that we are still lacking of enough knowledge to fully understand the physicochemical properties of peptides and the complex
interactions between peptides and stationary phase, which leads to the less-than optimum prediction of peptide RT. In terms
of the algorithm, the traditional model shows its limitation in tracing the many subtle factors that affect the peptide behaviors
on LC. Hence, in the field of peptide RT prediction, there is still large room for improvement.

Deep learning, an advanced machine learning method, has shown extraordinary capability to learn complex relationships
from large-scale data. There have been several tools that successfully utilized deep learning in RT prediction, such as
DeepRT~\cite{ma2018improved}, and Prosit~\cite{gessulat2019prosit}.

DeepRT use the capsule network (CapsNet)~\cite{sabour2017dynamic}
model. DeepRT could foresaw the RTs for the peptides at even different modification status included as oxidation of methionine, phosphorylation of
serine, threonine, tyrosine and at varied LC conditions included RPLC, SCX, HILIC.
Prosit utilizes the LTSM model and like the DeepRT, it could predict the retention time given the peptide sequence.
However, the DeepRT did not explain very well for the choice of CapsNet, and it did not fully consider the sequence's characteristics.
Prosit use the LSTM model to capture the this pattern. But, LSTM model has been beaten by the transformer model
in multiple tasks~\cite{vaswani2017attention}. Herein, we select the LSTM+transformer architecture.

\subsection{Ion Intensity Prediction}
Investigation of the peptide fragmentation is valuable both in theory and in practice. There are some researchers
focusing on the prediction of theoretical MS/MS spectra of peptides, including kinetic model-based methods and machine
learning based methods. MassAnalyzer~\cite{zhang2004prediction, zhang2005prediction} and MS-Simulator~\cite{sun2012ms,wang2015openms}
are two major kinetic model-based tools designed based on
the mobile proton hypothesis with some basic assumptions, and the key parameters of the models are tuned to fit the data
by statistics. The disadvantage of the kinetic model is that it cannot consistently be used to model the peptide
fragmentation under HCD, ETD, or electron-transfer and higher-energy collision dissociation (EThcD). PeptideART is
 a pure machine learning based tool which models the theoretical spectrum prediction as a classification problem, and
 the probability of the occurrence of each peak is learned by using a shallow feed-forward neural network~\cite{arnold2006machine,li2011accuracy}.
 Other previous work~\cite{frank2009ranking} predicts intensity ranks instead of relative intensities using learning-to-rank algorithms.
 It has been shown that a good prediction method can boost the identification of peptides. However, peptide
 fragmentation is very complex to predict; Li et al.~\cite{li2011accuracy} pointed out that the cross experiment correlations of PeptideART
 based on collision induced dissociation (CID) spectra were significantly lower than within-experiment analyses. To
 handle the complexity of peptide fragmentation, more powerful algorithms such as deep learning should be considered.

pDeep~\cite{zhou2017pdeep}, a deep learning-based method based on LSTM model to predict the intensity distribution of product ions of a peptide. pDeep can
work well in predicting not only HCD spectra but also ETD and EThcD spectra.pDeep achieved  $>$0.9 median PCCs (Pearson correlation coefficient)
in predicting HCD, ETD, and EThcD spectra, which is significantly higher than kinetic model-based MassAnalyzer and MS-Simulator as well
as the machine learning-based PeptideART. But, similarly like the RT prediction task, LSTM could not learn better than transformer, and this is why
we also choose the LSTM + transformer architecture for this task. After pDeep, the modified version of pDeep, called pDeep2~\cite{zeng2019ms}.

pDeep2 is spectrum predictor for modified peptides based on the deep learning model. It use the transfer-learning technique to transfer pDeep model
parameters to the prediction of modified peptide's spectrum. It claims that it's accurate model for predicting the spectra of peptides with common PTMs or low-abundance PTMs,
even if we only had a limited scale of benchmark modified PSMs.  Similarly, our model design make us to predict the spectrum of modified peptide.