
To train a better model, we sequentially pre-train the models in four datasets called 
HumanPhosDB~\cite{lawrence2016plug}, Jeff~\cite{liu2018vivo}, VeroE6~\cite{bouhaddou2020global}
and R2P2~\cite{leutert2019r2}. We split those pre-training datasets into training and validation set, selecting the best model on the validation set. The model is initialized by the selected model before training on the next pre-training dataset until those four datasets are all trained on. There are three downstream datasets called U2OS-DIA~\cite{wang2020naguider}, RPE1-DIA~\cite{bekker2020rapid} and RPE1-DDA~\cite{bekker2020rapid}. For the three downstream datasets, we manually set the $min(RT)$ and $max(RT)$ equals -100 and 200, respectively. -100 and 200 cold cover all the RTs in the three datasets, and the following researcher could directly use our well-trained model and the fixed $min(RT)$ and $max(RT)$ to predict the unknown RTs of their interested peptides.
Herein we use the square root of mean squared error (RMSE) as loss function.


Similarly to the RT task, we also use those ion intensity in three of four pre-training dataset Jeff~\cite{liu2018vivo}, VeroE6~\cite{bouhaddou2020global} and R2P2~\cite{leutert2019r2}, and fine-tune the pre-trained model on the downstream dataset U2OS-DIA~\cite{wang2020naguider}, RPE1-DIA~\cite{bekker2020rapid} and RPE1-DDA~\cite{bekker2020rapid}. The summary of the datasets used in this work is shown in Table~\ref{table:Datasets}.
We use mean squared
error (MSE) as our loss function. For both RT and Ion intensity tasks, we use Adam optimizer~\cite{kingma2017adam}, and the learning rate is 1e-4, the learning rate decay at the milestone epochs during the training. We implement our models by the Python and Pytorch, and train the model on multiple GPUs. 