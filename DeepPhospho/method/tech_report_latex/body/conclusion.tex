This study introduces DeepPhospho, a flexible deep neural network based on the LSTM-Transformer architecture that can predict retention times and tandem mass spectrometry spectra of peptides. It substantially surpasses current methods in several datasets. We are the first work that introduces the Transformer architecture into peptide chemico-physical properties prediction. Additionally, we design our method target on the phosphorylated peptides in the aspects of additional phosphorylated amino acid symbols, training loss of phosphorylated peptides, and transfer learning of pre-trained phosphorylation peptides datasets.

We believe that a more accurate RT and ion intensity prediction could benefit the downstream proteomics data investigation. For example, the researcher could use our well-trained model to build a predicted spectral library, which could hugely decrease the cost of the wet experiment but no loss of library accuracy. We could also fine-tune the model to adapt to a new type of library where the iteration would be much faster than the previous wet experiment method. 
