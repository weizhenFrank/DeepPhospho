In this study, we introduce DeepPhospho, a flexible deep neural network based on the LSTM-Transformer architecture able to predict retention times and tandem mass spectrometry spectra of peptides and it substantially surpasses current methods in several datasets. We are the first work which introduces the Transformer architecture into peptide chemico-physical properties prediction. Additionally, we design our method target on the phosphorylated peptides in the aspects of additional phosphorylated amino acid symbols, training loss of phosphorylated peptide and transfer learning of pre-trained phosphorylation peptides datasets.

We believe that a more accurate RT and ion intensity prediction could benefit the downstream proteomics data investigation. For example, researcher could use our well-trained model to build 
a predicted spectral libraries and this could hugely decrease the cost of wet experiment but no loss of accuracy of library. Also, we could fine-tuning the model to adapt to a new type of library which would be much faster iteration than the previous wet experiment method. 
