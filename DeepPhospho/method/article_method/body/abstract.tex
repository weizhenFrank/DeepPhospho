\begin{abstract}
   In mass-spectrometry-based proteomics, the identification and quantification of peptides and proteins heavily rely on sequence database searching or spectral library matching. The size and quality of the sequence database are crucial for the search. The research proposes we could build a virtual sequence database composed of the peptide sequence with its retention time and spectra by the computational method. Nowadays, the deep learning model has shown its power in computer vision and natural language processing. However, in proteomics, the lack of accurate predictive models for fragment ion intensities and retention time impairs the realization of the full potential of these proposals. Here, we propose our new method DeepPhospho, based on the LSTM-Transformer model, focusing on the phosphorylation peptide data, especially. By our model, researchers could build a larger and more accurate peptide library for protein identification in mass-spectrometry-based proteomics.
\end{abstract}