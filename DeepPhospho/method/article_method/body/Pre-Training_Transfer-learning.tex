\subsection{RT}
For RT prediction, to train a better model, we sequentially pre-train the models in four datasets called 
HumanPhosDB~\cite{lawrence2016plug}, Jeff~\cite{liu2018vivo}, VeroE6~\cite{bouhaddou2020global}
and R2P2~\cite{leutert2019r2}. We split those pre-training datasets into training : validation set = 9 : 1, respectively. We tune the hyper-parameters and select the best model on the validation set. The model is initialized by the selected model before training on the next pre-training dataset until those four datasets are all trained on. These pre-training RT dataset use their associated $min(RT)$ and $max(RT)$ to scale the raw RT value to 0-1. 

There are three downstream datasets called U2OS-DIA~\cite{wang2020naguider}, RPE1-DIA~\cite{bekker2020rapid} and RPE1-DDA~\cite{bekker2020rapid}. For the three downstream datasets, we manually set the $min(RT)$ and $max(RT)$ equals -100 and 200, respectively. -100 and 200 cold cover all the RTs in the three datasets, and the following researcher could directly use our well-trained model and the fixed $min(RT)$ and $max(RT)$ to predict the unknown RTs of their interested peptides. We load the pre-trained model as initialization for downstream dataset training. 

Herein we use the square root of mean squared error (RMSE) as loss function and Adam algorithm to optimize loss with learning rate 1e-3 for the first HumanPhosDB dataset, and 1e-4 for the other datasets. We decay the learning rate by 0.1 once the number of epoch reaches one of the milestones. And the milestones is manually selected by the learning curve during the training.

\subsection{Ion Intensity}
Similarly to the RT task, we also use three pre-training dataset Jeff~\cite{liu2018vivo}, VeroE6~\cite{bouhaddou2020global} and R2P2~\cite{leutert2019r2} to obtain a good initialization of ion intensity model and fine-tune the pre-trained model on the downstream datasets U2OS-DIA~\cite{wang2020naguider}, RPE1-DIA~\cite{bekker2020rapid} and RPE1-DDA~\cite{bekker2020rapid}. 

We use mean squared error (MSE) as our loss function and take the same training strategy including the optimizer and learning rate schedule as RT task.